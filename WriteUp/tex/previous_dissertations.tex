\documentclass[../main.tex]{subfiles} 
\begin{document}
  \section{Previous Dissertation Reviews}
    When reading previous dissertations, I tried to read a range of papers, to maximise their benefit to me.
    Firstly, I chose to read a paper that was strongly related to my own, \emph{`High Performance Computer Vision'} by Benjamin Eagan.
    This paper has the same topic,  Computer Vision, as my own.
    As such, I could learn a good amount about any hints or difficulties found with Computer Vision in the high performance domain.
    My next choice was \emph{`Parallelization and Optimization of a Tax and Benefits Model'} by Thomas McClintock. 
    This has almost no relation to Where's Wally, being a simulation of how taxes affect various groups of people.
    Thus, I could learn from this the general things that I should note in order to complete this project efficiently.
    However, my previous two papers were both on the list of dissertations which achieved a distinction.
    I therefore chose to read another paper, \emph{`Novel energy efficient compute architectures on the path to Exascale'} by Ioan Corneliu Hadade.
    From this paper, I could compare the differences between a paper that achieves a distinction, and one that achieves something between that and a pass.
    Hopefully I could extract that which makes the difference, and then attempt to do the same.
  \subsection{High Performance Computer Vision}
    Eagan's paper mostly deals with two topics; object tracking in real time videos, and disparity mapping.
    The first is useful for having more natural human-computer interactions.
    The second is for allowing computers to map 3D areas from 2D images.
    These are not the areas of computer vision that I will be concerning myself with, but are still in the general domain.
    
    The good qualities of the paper were apparent, presumably leading to the distinction grade it received.
    The quality of English throughout the paper was entirely understandable.
    The figures included were engaging, and generally it was easy to grasp what they were showing.
    The nature of the project that Eagan had chosen also strongly highlighted the importance of Risk Assessment.
    When choosing an algorithm to use, Eagan was very thorough in explaining why it was the optimum choice.

    However, I felt that the paper also had a few failings.
    The introduction section felt quite long, and took quite a while to get around to basic message of the section.
    This message was roughly ``Supercomputers are not used for Computer Vision because they are too expensive to run, but cheaper methods would be".
    I also felt that the explanation of the project goals was obfuscated within the lengthy introduction.
    Many figures were of processed images, but there were no figures of the original, making it hard to see the true effectiveness of the code.
    Some figures also showed errors in the processing.
    It would be useful if these captions briefly explained what caused these errors.
    Finally, some of the lines in the graphs included were very hard to distinguish from each other.
    The makes the plots somewhat indecipherable.  

    The paper also contained other things that I had not considered myself.
    These are neither good, nor bad, but their impact should be considered.
    \begin{itemize}
      \item The paper it written in something very close to default \LaTeX.
      \item Within the introduction, were standard definitions of speed-up and efficiency.
      \item Includes a list of figures and tables
      \item Each section is concluded with a brief summary of what came before
      \item Eagan chose to use the HSV colour scheme over the RGB scheme.
    \end{itemize}
  \subsection{Parallelization and Optimization of a Tax and Benefits Model}
  \subsection{Novel Energy Efficient Compute Architectures on the Path to Exascale}
  \subsection{Conclusions Drawn}
\end{document}
