\documentclass[../main.tex]{subfiles} 
\begin{document}
  \section{Risk Analysis}
    Listed here are some of the regular problems, followed by risks specifically related to this project.
  \subsection{Common Risks}
    \begin{description}
      \item[Slow Pace] (Probability: Possible, Impact: Severe)\\
        This happens when the target time for a project is underestimated, or when the project hits an unexpected snag.
        Firstly, slack time has been built into each section of the work plan, as seen in the work plan section.
        If the slow pace becomes critical, a simple version of the linear code will be used to produce a simple parallel version.
      \item[Fast Pace] (Unlikely, Negligible)\\
        In the unlikely situation work proceeds faster than expected, there are several areas for continued work.
        Firstly, optimisation of pre-existing programs, both linear and parallel, would be extremely useful.
        This will give definitive results about how much speed-up the program can obtain when recognising Wally in parallel.
        Secondly, a user interface that shows members of the public how the program works could be developed.
        This will start to create a project suitable for HPC Outreach.
      \item[In Absentia] (Possible, Moderate)\\
        Often in projects, a supervisor or student will be unable to attend meetings.
        A netbook is intended to be used regularly, enabling work to be done remotely.
        This extends to meetings, which can be done online using emails or other virtual techniques.
      \item[Broken Equipment] (Unlikely, Severe)\\
        If the normal workstation, a netbook, happens to break, then work can be continued on a desktop.
        This should be possible with the use of a remote versioning system, like github.
      \item[Sickness] (Possible, Moderate/Severe)\\
        Illness cannot be prevented, but in general, some amount of work should be able to be done.
        In the case of a major illness, the solution for slow pace will have to be implemented.
      \item[Ennui] (Certain, Moderate)\\
        Writers block will be solved by writing the dissertation report, or by implementing other sections of code.
        In the case of serious disillusionment, basic willpower will have to be exercised.
        
    \end{description}
   \subsection{Risks Specific to a Where's Wally Solver}
    \begin{description}
      \item[Unable to Find Wally] (Possible, Catastrophic)\\
        In the situation that my program cannot find Wally, the problem will have to be simplified.
        In testing various libraries, code has already been produced that can recognise a specific icon in an array of icons.
        Thus, in the simplest of cases, some code can work, and can then be parallelised.
      \item[Not Allowed To Use Wally] (Unlikely, Catastrophic)\\
        It is conceivable that  Walkers Books, publishers of Where's Wally would not allow me to use their Wally figure.
        Although this is legally dubious, there are ways around copyright laws.
        For example, it is unlikely they have copyrighted the concept of Where's Wally.
        This means a Where's Wally type puzzle could be generated.
        This could be done by randomly placing a large stock of images on a canvas.
      \item[Not Enough Wally Samples] (Probable, Moderate)\\
        This risk especially applies to any machine learning that might be done.
        The solution is to generate new puzzles, as with the above risk.
      \item[Not Fast Enough] (Probable, Negligible) \\
        If the program is not fast enough to compare to the speed that humans can solve a puzzle, then optimisations should be attempted.
        If this can't be done in a sufficient time frame, then the completion of a parallel solution will have to be the final stage of the project.
      \item[Not Suitable For HPC Outreach] (Probable, Negligible) \\
        This is very likely, and in this case, my project will have to be the starting point for another HPC Outreach project.
    \end{description}
\end{document}
