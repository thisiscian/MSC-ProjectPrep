\documentclass[../main.tex]{subfiles} 
\begin{document}
  \section{Introduction}
  Where's Wally is a series of puzzle books with a simple problem; you must find the eponymous Wally.
  He is wearing very distinct clothes; a pair of blue jeans, a red and white stripy jumper, with a woollen hat to match.
  Although he appears young, he carries a walking stick, and he wears glasses around his face.
  As such, viewed alone, he stands out as an entirely distinct.
  However, Wally appears alone only on the front cover of each book; within he is hidden behind a crowd of people and a plethora of objects.
  The human eye is a well honed tool; evolved over millions of years for the exact purpose of recognising objects visually.
  Even with such powerful tool, it can still take many minutes to find Wally.

  Computers on the other hand, have not spent millions of years developing object recognition schemes.
  Indeed, the origins of computer vision are rooted merely 40 years ago.
  Computers excel in logical operations, and recognising an object is not an immediately logical operation.
  Many factors change what an object looks like; perspective, lighting, objects in the background and foreground, etc.
  To a person, it is generally still obvious that a given object is the same object when these things change.
  It is not as simple for a computer.
  This computer must put in a lot of processing time to be able to recognise an object.
  Problems like Where's Wally are not trivial for it to solve, and they take a lot of time.

  Despite this, it is relatively easy to explain \emph{how} a program might solve a Where's Wally puzzle.
  Furthermore, parallelising the process should also be similarly easy.
  These two factors make a parallel solution of Where's Wally very alluring prospects for HPC Outreach.
  \biblio
\end{document}
