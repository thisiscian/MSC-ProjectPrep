\documentclass[main.tex]{subfiles} 
\begin{document}
  \section{Work Plan}
    As with most projects, there is a set time that I must complete the code and report within.
    According to the MSc. Programme Handbook\cite{url:msc-handbook}, the start of the dissertation period is on the 3rd of June 2013.
    The dissertation must be submitted by the 23rd of August 2013.
    This gives me 88 days to produce my project.

    My project, as presumably do many other \emph{`from scratch'} HPC projects, has two main regions areas of development: the linear code and the parallel code.
    As such it makes sense to divide my time into two regions, linear and parallel.
    This division is somewhat coarse, however, so each region will be divided into three new sections.

    The first will be \emph{Design}, wherein I will be deciding the overall format of the code.
    This will include, what programming language, libraries, functions and the actual method of solving Where's Wally.
    The next section will be \emph{Creating Tests}, where I will design the tests to ensure functionality of the solution.
    The final section will be \emph{Implementing the Solver}.
    This will be the actual production of code that satisfies the tests I have already created.

    If there is any spare time at the end of the project, a fourth section will be used; \emph{Optimisation and Outreach}.
    This will be simply improving the speed of the solution I have developed.
    If the program can solve Where's Wally problems at a near human level, then I will work on improving the output of the program.
    The end goal at this point, would be to produce something that could be used for HPC Outreach.
  \subsection{Breakdown}
    The breakdown of my work plan follows.
    Each section has built in time for weekends, plus some stretch for running overtime.
    \begin{description}
      \item[Linear (\textasciitilde30 days)]\hfill
        \begin{description}
          \item[Design (4 days)] As I am familiar with the problems of a Where's Wally solver, it is hoped that 4 days would be enough to design a solution.
          \item[Create Tests (6 days)] My problem lends itself to a relatively simple program, so testing shouldn't take more than a week.
          \item[Implement Solver (14 days)] A basic Where's Wally solver should be creatable within 2 weeks.
        \end{description}
      \item[Parallel (\textasciitilde40 days)]\hfill\\
        \begin{description}
          \item[Design (4 days)] Although a program utilising parallel methods is more complicated than a linear one, I hope to expand on the linear solution.
            Thus I have assigned the same amount of time as the linear design section.
          \item[Create Tests (6 days)] Again, I hope to reuse most of the tests from the linear section, so creating tests for the parallel version should take the same amount of time.
          \item[Implement Solver (21 days)] 
        \end{description}
    \end{description}
    The remaining 12 days will be entirely devoted to writing the dissertation report and presentation, however, it is intended that I will be updating these as I progress.
    This breakdown is visualised in Figure \ref{gantt-chart}, a Gantt chart.
    
    \subfile{gantt}
    \biblio
\end{document}
