\documentclass[main.tex]{subfiles} 
\begin{document}
  \section{Work Plan}
    According to the MSc. Programme Handbook\cite{url:msc-handbook}, the start of the dissertation period is on the 3rd of June 2013.
    The dissertation must be submitted by the 23rd of August 2013.
    This means that there are 88 days to start and complete the project.

    This project has two main strands of development: the linear code and the parallel code.
    This division is somewhat coarse, however, so each region will be divided into three new sections.

    The work plan is broken down below.
    Note that each section has built in time for weekends, plus some stretch for running overtime.
    Furthermore, the text inside the square brackets (e.g. [L1]) indicates the path on the Gantt chart.
    This can be found in the appendix.
    \begin{description}
      \item[Linear (\textasciitilde30 days)]\hfill
        \begin{description}
          \item[Designing {[L1]} (4 days)]
            This will consist of deciding the format of the linear code.
            This includes choosing a programming language and library to use.
            Basic function concepts will also be developed here.
            It is hoped that 4 days would be enough to design this solution.
          \item[Testing {[L2]} (6 days)]
            This project will be test driven.
            As such, creating tests comes before any implementation stages.
            This time will be spent devising and implementing tests to ensure that the program works correctly.
            The problem is conceptually simple and it is hoped that the tests will reflect this.
            Six days have been assigned to creating the tests.
          \item[Implementation {[L3]} (14 days)]
            Working from the basis of the tests, code will be written that satisfies the tests.
            Once this has happened, a working solution for the Where's Wally puzzle will have been created.
            Fourteen working days have been given to implementing the solution.
          \item[Optimisation {[L4]} (any remaining time)]
            If the linear and parallel programs have both been produced, then spare time will be used optimising both solvers.
            This is important, as it is the main way to benchmark any speed-up that the parallel code exhibits.
            Optimising the linear program will likely involve reducing memory access, and so time will be focused on this.
        \end{description}
      \item[Parallel (\textasciitilde40 days)]\hfill
        \begin{description}
          \item[Designing {[P1]} (4 days)]
            Although a program utilising parallel methods is more complicated than a linear one, this project intends to expand on the linear solution.
            This means that functions from the linear program will be analysed for potential parallelism.
            Four days have been put aside for this stage.
          \item[Testing {[P2]} (6 days)]
            Here, new tests will have to be written to ensure that the parallelised functions designed above work correctly.
            The old tests should hold true however, so only a small amount of testing needs to be done.
            Creating specific tests for the parallel version should 6 days.
          \item[Implementation {[P3]} (21 days)] 
            The implementation of the parallel program consists of satisfying the tests laid out during the testing period.
            It is intended that this will mostly be through the modification of existing code.
            As parallel programming is more difficult than linear program, implementing a parallel solution has been given 3 weeks.
          \item[Optimisation {[P4]} (any remaining time)]
            Again, any time remaining at the end of the project will be devoted to optimising the solver.
            The goal is to have the parallel run as fast, if not faster, than a human solving the same puzzle.
        \end{description}
      \item[Dissertation Report (ongoing)]\hfill
        \begin{description}
          \item[Writing Report {[D1]} (ongoing)]
            The writing of the report can be begin as soon as any amount of design has been done.
            The report will be updated as work continues, ensuring that the report is never a large workload.
            This has the added benefit of allowing more time for editing and correction.
            During this time, the project presentation will also be written.
          \item[Present Dissertation {[D2]} (2 days)]
            Over two days, the dissertation presentations will be held.
            Time between the submission date and the presentation of this report will be spent practising the presentation.
        \end{description}
    \end{description}
    \biblio
\end{document}
