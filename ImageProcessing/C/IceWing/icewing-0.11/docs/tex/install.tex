\chapter{Short overview}

\paragraph{What's all about?}\hfill\\
\index{iceWing}
\icewing{}, an {\bf I}ntegrated {\bf C}ommunication {\bf E}nvironment
{\bf W}hich {\bf I}s {\bf N}ot {\bf G}esten\index{Gesten} (This is
a reference to an older program, the predecessor of \icewing{}.) is
a graphical plugin shell. It is optimized for image processing and
vision system development. But its use in totally different fields
of research is as well possible, e.g. audio-stream processing.
Predefined or self-written plugins operate hierarchically on data
provided by other plugins and can also generate new data-streams,
allowing flexible communication and interaction between these
plugins. An important predefined plugin is the grabbing plugin,
which can read images from the disk in various image formats, from
grabber-hardware, e.g. V4L2-devices or FireWire, and also from
external, network wide processes via DACS streams.

Not being only a batch-plugin-shell \icewing{} is also a highly
customizable GUI platform for the plugins: It has a list of given
GUI-elements and allows the plugins to simply make use of them. So
plugins can show the user their current status and can let the user
change parameters on the fly. Moreover methods of easy visualization
of plugin results are available. The plugins can open any number of
windows and display in these windows any data in a graphical form.

\paragraph{Where to get \icewing{}?}
\begin{itemize}
\item \index{iceWing!homepage} The homepage for \icewing{} can be found at\\
  \hspace*{4ex}\url{http://icewing.sourceforge.net}
\item \index{iceWing!mailing lists} There are two mailing lists
  dedicated to iceWing user and development discussion. More infos
  about these mailing lists can be found at\\
  \hspace*{4ex}\url{http://sourceforge.net/mail/?group_id=151242}\\
  Please use these lists for any questions, patches, or anything
  else concerning \icewing{}.
\item Additionally, the main author can be reached at\\
  \hspace*{4ex}floemker@techfak.uni-bielefeld.de
\end{itemize}

\paragraph{Who did \icewing{}?}\hfill\\
Program:\\
  \hspace*{4ex}Frank L\"omker, floemker@techfak.uni-bielefeld.de\\
This documentation:\\
  \hspace*{4ex}Frank L\"omker, floemker@techfak.uni-bielefeld.de\\
  \hspace*{4ex}Initial installation and user guide: Andreas H\"uwel, andreas.huewel@gmx.de\\
  \hspace*{4ex}Programming guide translation of V0.8.1: Ilker Savas

\paragraph{License}\hfill\\
\icewing{} is free software; you can redistribute it and/or modify
it under the terms of the GNU General Public License as published by
the Free Software Foundation; either version 2 of the License, or
(at your option) any later version.

\icewing{} is distributed in the hope that it will be useful,
but WITHOUT ANY WARRANTY; without even the implied warranty of
MERCHANTABILITY or FITNESS FOR A PARTICULAR PURPOSE. See the
GNU General Public License for more details.

You should have received a copy of the GNU General Public License
along with this program; if not, write to the Free Software
Foundation, Inc., 59 Temple Place, Suite 330, Boston, MA 02111-1307
USA.

\chapter{\Index{Installation}}
\label{chap:installation}

\section{Requirements}

\paragraph{Libraries needed}

\icewing{} is targeted at Unix-like operating systems and was
actually tested on Linux, Alpha/True64, Solaris, and Mac OSX. For
compiling and using it some programs and libraries must be
installed:

\begin{itemize}
\item Basic commands like bunzip2, tar, make, C compiler
  (probably gcc) - nothing that a normal Unix installation does not
  provide.
\item X11
\item \Index{GTK+}: For its interface \icewing{} can use Version 1.2
  (tested for version \textgreater=1.2.5) as well as version V2.x of
  GTK+ (see ``\url{http://www.gtk.org}'').
\item \Index{CMake}: For configuring and for generating build files
  CMake can be used (see ``\url{http://www.cmake.org}'').
  Alternatively, there are hand written makefiles, which can be
  adopted to the respective environment.
\end{itemize}

Additional \emph{optional} libraries extend the functionality of
\icewing{}:
\begin{itemize}
\item \Index{gdk-pixbuf}: Enables loading of additional image
  formats besides PNM and PNG. This library is only needed if you
  compile against GTK+ V1.2. Since GTK+ V2.0 this library is
  integrated into GTK and thus always automatically
  available. Versions \textgreater= 0.8 where tested.
\item \Index{libjpeg}: Enables JPEG image saving and loading and
  saving of AVI/MJPEG movies.
\item \Index{libpng}: Enables PNG-saving and loading of PNG images
  with a depth of 16 bit.
\item \Index{libz}: Is needed by the above optional PNG library.

  When the jpeg and png libraries are not available, \icewing{} can
  only save the various PNM formats and AVI movies with a raw
  uncompressed codec.
\item \Index{libAVVideo}: On Alpha/True64 systems (and only there)
  the libAVVideo library is needed to access grabber hardware. This
  library is written by the AG-AI and supports grabbing of images
  via the Multi Media Extension MME. Composite and SVideo cameras
  are supported.
\item \Index{libraw1394}, \Index{libdc1394}\label{page:libdc1394}:
  These two libraries enable grabbing from FireWire (IEEE1394)
  cameras supporting the ``Digital Camera Specification''. See
  ``\url{http://ieee1394.wiki.kernel.org}'' and
  ``\url{http://sourceforge.net/projects/libdc1394/}'' for the
  libraries. Version 2 of the libdc1394 library as well as version
  1.2 is supported.
\item \Index{Aravis library}\index{libaravis}: Aravis is a library
  for video acquisition using \Index{GenICam} cameras
  (``\url{http://www.genicam.org}''). It currently implements an
  ethernet camera protocol used for industrial cameras. See
  ``\url{http://live.gnome.org/Aravis}'' for details about this
  library. Versions 0.1.14 and 0.1.15 were tested.
\item \Index{NET iCube} NETUSBCAM: This library enables
  grabbing images from cameras supporting the interface from
  NET. See ``\url{http://www.net-gmbh.com}'' for the driver and the
  cameras. Versions 1.23 till 1.25 were tested.
\item \Index{MATRIX VISION} MV impact acquire: This library enables
  grabbing images from cameras supporting the interface from MATRIX
  VISION. See ``\url{http://www.matrix-vision.com}'' for the driver
  and the cameras. Versions from 1.12.44 till 2.0.5 were tested.
\item \Index{IDS Imaging} uEye: This library enables grabbing images
  from cameras supporting the interface from IDS Imaging. See
  ``\url{http://www.ids-imaging.com}'' for the driver and the
  cameras. Versions 3.90 and 4.00 were tested.
\item \Index{unicap}: Unicap provides a uniform API for different
  kinds of video capture devices, e.g. IEEE1394, Video for Linux,
  and some other. See ``\url{http://unicap-imaging.org}'' for
  details about this library. Version 0.9.14 was tested.
\item \Index{libv4lconvert}: libv4lconvert from the v4l-utils
  package provides conversion functions for different additional
  pixel formats for the V4L2 grabbing driver. See
  ``\url{http://git.linuxtv.org/v4l-utils.git}'' for details about
  this library. Versions 0.8.6 till 0.8.8 were tested.
\item \Index{FFmpeg}: Enables loading of a wide variety of video
  formats and saving of AVI/FFV1 and AVI/MPEG-4 movies. See
  ``\url{http://www.ffmpeg.org}'' for further details about this
  library. Without the FFmpeg library, \icewing{} can only handle
  AVI files with the motion jpeg codec (if the jpeg library is
  available) and AVI files with RAW images in the YVU444P or YUV420P
  color spaces.

  During development of \icewing{} different released versions
  starting from version 0.5 (3. of March, 2009) till version 1.0
  (28. of September, 2012) and different SVN and GIT snapshots till
  1. of October, 2012 where used. You can find the FFmpeg SVN
  snapshot version from June 2009 at the \icewing{} homepage. Older
  versions, especially the old release version 0.4.9-pre1 are too
  old and will \emph{not} work.
\end{itemize}

\emph{Attention:} You will always need the development files,
especially header files, for the different libraries. We will show
for RedHat packages (rpm) how to verify that everything is
installed. Debian packages are quite similar. You can easily check
with the following shell command
\sS
\shellcom{rpm -qa | grep gtk}
\sE
which packages are installed and which version that libraries have.
\icewing{} needs the headers, too, and often separate developer packets
provide them. If you compile your own libraries from sources, you
have to add the headers into the default-include path, so \icewing{}
will find them. After verifying all this, you only need to get the
\icewing{} tarball ``icewing-version.tar.bz2''.

\paragraph{Further stuff}

You can use \index{DACS}\dacs{} for your own \icewing{} plugins to
communicate with other external net wide processes. It works quite
similar to Corba. You do {\em not} need \dacs{} for any \icewing{}
internal communication (\icewing{} - plugin, plugin - plugin) or to
e.g. access files of the system. Additionally, \icewing{} has
integrated support for reading/publishing images to/from other
programs and for remote control of plugin sliders via \dacs{}. If
\dacs{} is not available, these features can be disabled during
compiling.

More information about \dacs{} can be found in the Web under this
address:
\begin{quotation}
\url{http://www.techfak.uni-bielefeld.de/ags/ni/projects/dacs/}
\end{quotation}
Many details of \dacs{} are also in the dissertation of Niels
Jungclaus \cite{Jungclaus1998-IVS}.


%######################
\section{Installation}

Lets assume you have the \icewing{} tarball in ``./''. Then simply
start by
\sS
\shellcom{bunzip2 -c icewing-version.tar.bz2 | tar -xv}
\shellcom{cd icewing-version}
\sE
where version is the particular \icewing{} version number you are
using.

To proceed, you have two distinct possibilities, as there are two
installation procedures available: Using CMake to configure
\icewing{} and generate necessary build files or to adapt and use
the hand written makefiles.

\subsection{Using \Index{CMake} for \icewing{} building}

CMake, available from \url{http://www.cmake.org}, is a utility to
configure a software package and generate build files for it. If you
would like to use it for \icewing{}, verify that you have CMake
installed. After that go to the cmake directory inside the
\icewing{} sources and start the cmake utility, i.e.:
\sS
\shellcom{cd \{wherever your sources are\}/cmake}
\shellcom{cmake ..}
\sE
This will generate the necessary build files for building \icewing{}
in the cmake subdirectory and for installing it to
``/usr/local''. Compiling and installing is then done by
\sS
\shellcom{make}
\shellcom{make install}
\sE
inside the cmake subdirectory. Optionally, ``\verb|VERBOSE=1 make|''
will show the exact commands that are run during the make
process. ``\verb|make install|'' will create a file
``\verb|install_manifest.txt|'' in the cmake directory, which
contains the list of all installed files.

Additionally, you can pass different variable settings to CMake with
the ``-D'' command line to influence the configuration step, i.e.:
``\verb|cmake -DNAME_OF_THE_VARIABLE=value ..|''. Important
available variables are:
\begin{description}
\item[CMAKE\_INSTALL\_PREFIX=PATH:] The installation prefix, e.g.
  ``\verb|cmake -DCMAKE_INSTALL_PREFIX=/opt/icewing ..''| will
  install \icewing{} to ``/opt/icewing''. The default is
  ``/usr/local''.
\item[WITH\_GRABBER=ON\textbar{}OFF:] If OFF, \icewing{} will have
  \emph{no} support for grabbing images from a camera, independent
  of any installed libraries. Otherwise, depending on the found
  libraries, support for different grabber hardware will be
  integrated. The default is ON.

  On Alpha/True64 systems the libAVVideo is needed for all grabber
  access. On Linux, support for Composite and SVideo cameras is
  based on the ``Video for Linux Two'' interface (see
  ``\url{http://linuxtv.org/wiki/}'') and directly integrated in
  \icewing{}. If available, the libv4lconvert library (see
  ``\url{http://git.linuxtv.org/v4l-utils.git}'') is used to support
  different additional pixel formats, pixel formats which are
  sometimes used be web cams.

  FireWire (IEEE1394) cameras supporting the ``Digital Camera
  Specification'' are supported via the raw1394 and dc1394
  libraries. If available version 2 of the dc1394 library is used,
  otherwise version 1.2 is used. Cameras conforming to the NET iCube
  interface are supported if the iCube NETUSBCAM library can be
  found. Cameras conforming to the MATRIX VISION impact acquire
  interface are supported if the impact aquire libraries can be
  found. Cameras conforming to the IDS Imaging uEye interface are
  supported if the uEye libraries can be found. Cameras using the
  GenICam standard are supported by the aravis library. Additional
  hardware is supported if the unicap library is available.
\item[WITH\_GTK1=ON\textbar{}OFF:] If ON, force using GTK+ V1.2 for
  the user interface. If OFF, first try to find GTK+ V2.x and fall
  back to GTK+ V1.2 if V2.x can not be found. The default is OFF.
\item[CMAKE\_BUILD\_TYPE=STRING:] Type of build. The possible
  choices are: None (CMAKE\_CXX\_FLAGS and CMAKE\_C\_FLAGS are
  used), Debug, Release, RelWithDebInfo, and MinSizeRel. The default
  is ``RelWithDebInfo'', which uses ``-O2 -g'' if gcc is used as the
  compiler.
\end{description}

CMake will try to find all necessary and supported libraries
automatically. If you want to disable this and prevent the usage of
a library, you can do so by using one of the following variables:
\begin{description}
\item[WITH\_DACS=ON\textbar{}OFF:] Build with DACS support if DACS is installed.
\item[WITH\_DOC=ON\textbar{}OFF:] Build the iceWing PDF manual.
\item[WITH\_XCF=ON\textbar{}OFF:] Build XCF communication support.
\item[WITH\_PNG=ON\textbar{}OFF:] Build PNG image saving support.
\item[WITH\_JPEG=ON\textbar{}OFF:] Build JPEG image saving support.
\item[WITH\_ARAVIS=ON\textbar{}OFF:] Build aravis grabbing support.
\item[WITH\_FIRE2=ON\textbar{}OFF:] Build Firewire grabbing support.
\item[WITH\_ICUBE=ON\textbar{}OFF:] Build NET iCube grabbing support.
\item[WITH\_UNICAP=ON\textbar{}OFF:] Build unicap grabbing support.
\item[WITH\_LIBV4L=ON\textbar{}OFF:] Use libv4lconvert in the V4L2 grabbing driver.
\item[WITH\_MVIMPACT=ON\textbar{}OFF:] Build MATRIX VISION mvDeviceManager grabbing support.
\item[WITH\_UEYE=ON\textbar{}OFF:] Build IDS uEye grabbing support.
\item[WITH\_FFMPEG=ON\textbar{}OFF:] Build extended movie loading/saving support.
\item[WITH\_READLINE=ON\textbar{}OFF:] Build icewing-control command line tool.
\end{description}

If you have headers and libraries installed in
non-standard locations that CMake cannot find, then set the
following two environment variables. Despite the similar naming
convention, these will not work as arguments on the cmake command
line:
\begin{description}
\item[CMAKE\_INCLUDE\_PATH:] E.g. export CMAKE\_INCLUDE\_PATH=/sw/include
\item[CMAKE\_LIBRARY\_PATH:] E.g. export CMAKE\_LIBRARY\_PATH=/sw/lib 
\end{description}

After the ``make install'', if the new ``icewing''-executable is in
the execute-path, you can right away start ``icewing'', the
executable (see section~\ref{sec:quicktour} ``quicktour''). If you
are interested in what is where in this installation, have a look at
section~\ref{sec:p_filesystem} ``filesystem hierarchy''.

\subsection{Using makefiles for \icewing{} building}

Alternatively to the CMake build system, hand written makefiles are
available for compiling the \icewing{} package. To use these, you have
to edit the \Index{Makefile} to adopt it to your system and in- or
exclude the support for additional packages. Mostly it will be
uncommenting some few lines that you probably will not need on your
system.

\begin{description}

\item[PREFIX:]
  The installation prefix, i.e. where \icewing{} will be
  installed during ``make install''. The default is ``/usr/local''.

\item[FLAGS:]
  Eventually you must adopt the compiler flags to your
  hardware. E.g.: You might not have a Pentium/Athlon processor?

\item[WITH\_GTK1:]
  If WITH\_GTK1 is defined GTK+ V1.2 is used for the user
  interface. Otherwise \icewing{} uses the newer 2.x GTK+ version.

\item[WITH\_GRABBER, WITH\_ARAVIS, WITH\_FIRE,]
  \textsf{\textbf{WITH\_ICUBE, WITH\_MVIMPACT, WITH\_UEYE,
      WITH\_UNICAP, WITH\_LIBV4L:}}
  If included, \icewing{} will have support for grabbing images from
  a camera. On Alpha/True64 systems an external library -- the
  libAVVideo -- is needed. The value given to WITH\_GRABBER
  specifies the location of the libAVVideo.

  On Linux support for image grabbing is directly integrated into
  \icewing{}. This gets activated if WITH\_GRABBER is
  defined. Composite and SVideo cameras are supported with the help
  of the ``Video for Linux Two'' interface (see
  ``\url{http://linuxtv.org/wiki/}''). If WITH\_LIBV4L is defined
  the libv4lconvert library (see
  ``\url{http://git.linuxtv.org/v4l-utils.git}'') is used to support
  different additional pixel formats, pixel formats which are
  sometimes used be web cams. WITH\_LIBV4L gives the location of the
  libv4lconvert library and header files.

  If additionally WITH\_FIRE is defined FireWire (IEEE1394) cameras
  supporting the ``Digital Camera Specification'' are supported as
  well. WITH\_FIRE gives the location of the needed raw1394 and
  dc1394 libraries and header files, for example ``/usr'' or
  ``/usr/local''. If WITH\_FIRE2 is defined, version 2 of the dc1394
  library is used instead of version 1.2. WITH\_ICUBE activates
  support for cameras conforming to the NET iCube
  interface (see ``\url{http://www.net-gmbh.com}''). WITH\_ICUBE
  gives the location of the iCube NETUSBCAM library and header
  files. WITH\_MVIMPACT activates support for cameras conforming to
  the MATRIX VISION impact acquire interface (see
  ``\url{http://www.matrix-vision.com}''). WITH\_MVIMPACT gives the
  location of the impact aquire libraries and header
  files. WITH\_UEYE activates support for cameras conforming to the
  IDS uEye interface (see
  ``\url{http://www.ids-imaging.com}''). WITH\_UEYE gives the
  location of the uEye libraries and header files. WITH\_ARAVIS
  additionally to WITH\_GRABBER activates support for the aravis
  library, which supports cameras conforming to the GenICam standard
  (see ``\url{http://live.gnome.org/Aravis}''). WITH\_ARAVIS gives
  the location of the aravis library and header files. By defining
  WITH\_UNICAP additionally to WITH\_GRABBER support for the unicap
  library gets integrated. WITH\_UNICAP gives the location of the
  unicap library and header files (see
  ``\url{http://unicap-imaging.org}'').

\item[WITH\_DACS:]
  If included, \icewing{} will have support for \dacs{}, which is
  like Corba for network wide interprocess
  communication. Especially, \icewing{} will be able to send and
  receive images and widget configurations via \dacs{} communication
  channels.

\item[WITH\_GPB:]
  Enables support for the gdk-pixbuf library to load images of other
  formats than PNM and PNG.

\item[WITH\_JPG, WITH\_PNG, WITH\_ZLIB:]
  Enables further image saving formats and enhances image loading by
  using the libjpeg, libpng, and libz libraries.

\item[WITH\_FFMPEG:]
  Enables loading of a wide variety of video formats and saving of
  two additional video formats by using the FFmpeg
  library. WITH\_FFMPEG gives the location of the library and the
  corresponding header files.
\end{description}

\paragraph{``make'' it all}
After adopting the Makefile, you can build the installation files
from the sources:
\sS
\shellcom{cd \{wherever your sources are\} }
\shellcom{make depend}
\shellcom{make}

You must now login as admin/root, if \$\{PREFIX\} is not writable for
the current user. The Makefile expects the directory that is named
in \$\{PREFIX\} to be existing - if not:
\sS
\shellcom{mkdir \$\{PREFIX\}}
\sE
The installation is now fully prepared. Now the time for 
installation has come!
\sS
\shellcom{cd \{wherever your sources are\}}
\shellcom{make install}
\sE
To all the cautious admins: 
You eventually want to check the groups and rights of the new dirs
now.

{\em That's it!}
For further details about what was done, just have insight into the
installation log file, which is located at
``\$\{PREFIX\}/share/log/iceWing.log''.

If the new ``icewing''-executable is in the execute-path, you can
right away start ``icewing'', the executable (see
section~\ref{sec:quicktour} quicktour). If you are interested in what is
where in this installation, have a look at
section~\ref{sec:p_filesystem} Filesystem.

\paragraph{Troubleshooting}

Check the output for errors, also the installation-logfile. You are
sure, that you installed the needed libraries properly, but
\sS
\shellcom{make}
\sE
produced errors like:
\begin{quote}
/bin/sh: gdk-pixbuf-config: command not found
\end{quote}

Let's again assume you used the RedHat package named
``gdk-pixbuf-devel'' (well, Debian packages are treated similar).
Then check via
\sS
\shellcom{rpm -ql gdk-pixbuf-devel | grep gdk-pixbuf-config}
\sE
or with {\em full} search
\sS
\shellcom{find / -name gdk-pixbuf-config}
\sE
{\em where} the needed packed-files got installed.
Maybe they are simply not in the default-execute path!?

Assume you find ``gdk-pixbuf-config'' 
in ``/opt/gnome/bin/gdk-pixbuf-config''. With the bash-shell
you can replace the compiling 
\sS
\shellcom{make}
\sE
command by:
\sS
\shellcom{PATH=\$PATH:/opt/gnome/bin make}
\sE
Or you change the Makefile entry ``GDK\_PIXBUF = gdk-pixbuf-config''
to ``GDK\_PIXBUF = /opt/gnome/bin/gdk-pixbuf-config''.

%%% Local Variables: 
%%% mode: latex
%%% TeX-master: "../iceWing"
%%% fill-column: 68
%%% End: 
